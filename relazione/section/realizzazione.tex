\section{Realizzazione}
Ogni pagina ha un suo scheletro in un file HTML, il quale viene aperto dal file \textit{nome\textunderscore pagina.php} tramite la funzione \textit{file\textunderscore get\textunderscore contents()}, successivamente la pagina viene popolata dalla classe PHP \textbf{htmlMaker} da noi sviluppata e solo a questo punto viene mostrata all'utente.\\
Così facendo viene mantenuto totalmente separato il comportamento dalla struttura, rimane possibile gestire le sessioni e non sono necessarie chiamate AJAX dalle pagine.
\subsection{HTML}
Abbiamo cominciato scrivendo un header ed un footer che fossero unici per tutte le pagine, questi due file HTML vengono letti dalla classe \textit{htmlMaker} e inseriti in tutte le pagine presenti nel sito. Successivamente abbiamo iniziato a scrivere il codice HTML di ogni altra pagina, a partire dalla \textit{home}, \textit{prodotti} e \textit{contatti}. Man mano che le pagine prendevano forma andavamo a spostare il codice HTML scritto nella classe PHP responsabile della generazione del contenuto delle varie pagine.
\subsection{CSS}
Riguardo allo stile CSS, abbiamo lavorato parallelamente allo sviluppo delle pagine a partire dallo stile principale condiviso per poi procedere anche con lo stile per dispositivi mobile, la quale presentazione è gestita da un apposito foglio di stile che ottimizza la visualizzazione e dispone barra di navigazione e pulsanti di ricerca e account nel menu ad hamburger, situato nella parte destra dell'header.


\begin{figure}[H]
	\centering
	\includegraphics[width=6cm]{utility/home_mobile.png}
	\caption{Il layout mobile della homepage}
\end{figure}

\pagebreak

Infine è stato predisposto un foglio di stile per la stampa, dalla quale sono stati nascosti alcuni elementi giudicati superflui (ad esempio la barra di navigazione o i pulsanti back-to-top, ricerca e account) o modificati per migliorarne la presentazione (come la breadcrumb, le recensioni o la grandezza delle immagini principali).

\begin{figure}[H]
	\centering
	\includegraphics[width=16cm]{utility/prodotti_printcss.png}
	\caption{Il layout di stampa della pagina Prodotti}
\end{figure}

I colori dominanti del sito sono il grigio scuro ed un verde pistacchio per header, sfondi ed altri elementi di presentazione, mentre vengono utilizzati il nero, il verde ed il giallo per testi e link.\\

% palette colori
Sfondi ed elementi
\definecolor{hdgray}{RGB}{34, 34, 34}
\definecolor{pistacho}{RGB}{192, 219, 181}
\definecolor{puffo}{RGB}{212, 230, 247}
\crule[hdgray]{1cm}{1cm} \crule[pistacho]{1cm}{1cm} \crule[puffo]{1cm}{1cm} 


Testi e link
\definecolor{bcgreen}{RGB}{167, 255, 131}
\definecolor{hdgreen}{RGB}{216, 251, 216}
\crule{1cm}{1cm} \crule[yellow]{1cm}{1cm} \crule[bcgreen]{1cm}{1cm} \crule[hdgreen]{1cm}{1cm}
\\

La palette è stata scelta in modo da garantire una corretta visualizzazione anche da persone affette da diversi tipi di cecità o sensibilità del contrasto ed il contenuto del sito rimane correttamente visualizzabile anche senza il caricamento dello stile CSS.

\begin{figure}[H]
	\centering
	\includegraphics[width=16cm]{utility/prodotti_deuteranopia.png}
	\caption{Vista della pagina Prodotti da persone affette da deuteranopia (cecità al verde)}
\end{figure}


\subsection{Javascript}
Tutta la parte di JavaScript del progetto è contenuta in un unico file \verb|main.js| e viene invocata da un link esterno nella sezione head di ogni pagina. \\
Il file \verb|main.js| è stato come segue strutturato.
\paragraph{Validatori}
I primi metodi servono per controllare la corretta compilazione dei vari campi di input testuali presenti nelle form di registrazione di un account e di modifica dei dati di un utente. Inoltre nel caso di non corrispondenza con la specifica grammatica in seguito definita si occupano di mostrare lo specifico errore.
\begin{itemize}
    \item \textit{validateName(str)}: controlla che il nome dell'utente corrisponda all'espressione regolare \verb| /^[A-Z ,.'-]{3,30}$/i|. Saranno accettate quindi solamente stringhe di lunghezza compresa tra i 3 e i 30 caratteri, composte da caratteri alfabetici e i caratteri \verb|. , ' -|;
    \item \textit{validateSurname(str)}: controlla che il cognome dell'utente corrisponda all'espressione regolare\verb| /^[A-Z ,.'-]{3,30}$/i|. Saranno accettate quindi solamente stringhe di lunghezza compresa tra i 3 e i 30 caratteri, composte da caratteri alfabetici e i caratteri \verb|. , ' -|;
    \item \textit{validateUsername(str)}: controlla che lo username dell'utente corrisponda all'espressione regolare\verb| /^[w_]{4,30}$/i|. Saranno accettate quindi solamente stringhe di lunghezza compresa tra i 4 e i 30 caratteri composte da caratteri alfabetici e il carattere \verb|_| ;
    \item \textit{validatePassword(str)}: controlla che la password dell'utente corrisponda all'espressione regolare\verb| /^[w_]{4,30}$/i|. Saranno accettate quindi solamente stringhe di lunghezza compresa tra i 4 e i 30 caratteri composte da caratteri alfabetici e il carattere \verb|_| ;
    \item \textit{validateEmail(str)}: controlla che l'email dell'utente corrisponda all'espressione regolare\begin{verbatim}/[\S]{4,32}@[\w]{4,32}((?:\.[\w]+)+)?(\.(it|com|edu|org|net|eu)){1}/\end{verbatim}Le stringhe accettate dovranno quindi essere composte da una stringa tra i 4 e i 32 caratteri come identificativo, il carattere di at \verb|@|, il dominio tra i 4 e i 32 caratteri e infine un suffisso tra \texttt{(.it|.com|.edu|.org|.net|.eu)}.
\end{itemize}
\paragraph{Controlli input}
I metodi che seguono vengono utilizzati dalle invocazioni tramite submit delle form.
\begin{itemize}
    \item \textit{CheckRegistrazione()}: si occupa di recuperare il valore dei campi della form di registrazione di un nuovo utente e restituisce true se tutti i campi sono stati compilati e se rispettano le grammatiche prefissate;
    \item \textit{CheckModificadati()}: si occupa di recuperare il valore dei campi della form di modifica delle informazioni di un utente, restituisce true se tutti i campi sono stati compilati e se rispettano le grammatiche prefissate;
\end{itemize}
\paragraph{Gestori presentazione}
Gli ultimi tre metodi servono solamente per la modifica della contenuto del DOM anche al fine di una migliore portabilità verso gli schermi più piccoli.
\begin{itemize}
    \item \textit{getSearchbar()}: Questo è un metodo trigger che viene invocato nel caso di click sull'icona di ricerca, si occupa di rendere visibile la barra di ricerca ed assegnarle il focus. Nel caso di un secondo click la barra tornerà invisibile;
    \item \textit{checkWidth()}: Questo è un metodo trigger che viene invocato al caricamento di ogni pagina e al rilevamento di una modifica alla larghezza del body. Quando la larghezza del display diventa inferiore a \verb|850px|, come in un dispositivo mobile, si occupa di rimuovere le icone, la barra di navigazione e la barra di ricerca dalla parte di header visibile e aggiungerle al menù laterale a comparsa.
    Viceversa, nel caso in cui la larghezza dovesse superare \verb|850px| riposiziona i contenitori nella parte di header visibile ;
    \item \textit{showMenu()}: Questo è un metodo trigger che viene invocato nel caso di click sull'icona del menù ad hamburgher. Si occupa di mostrare in primo piano il menù nascosto con l'applicazione o la rimozione della classe di stile \verb|open|;
\end{itemize}

\subsection{PHP}
Ogni pagina del sito è generata da un file PHP.
Abbiamo creato due classi con metodi statici, una per la costruzione dell'HTML e una per la connessione al database, in modo da non aver bisogno di istanziare oggetti per richiamarne le funzioni:
\begin{itemize}
\item \textbf{htmlMaker.php:} si occupa della costruzione di codice HTML, ed é composta dai seguenti metodi:
	\begin{itemize}
		\item \textit{makeHead}: metodo utilizzato da tutte le pagine del sito, costruisce l'head HTML.
		\item \textit{makeHeader}: metodo utilizzato da tutte le pagine del sito, costruisce l'intestazione HTML.
		\item \textit{makeHeading}: metodo utilizzato da tutte le pagine del sito, costruisce il titolo principale.
		\item \textit{makeBreadCrumbs}: metodo utilizzato da tutte le pagine del sito, costruisce la breadcrumb.
		\item \textit{makeTornaSu}: metodo utilizzato da tutte le pagine del sito, costruisce il pulsante per tornare all'inizio della pagina.
		\item \textit{makeFooter}: metodo utilizzato da tutte le pagine del sito, costruisce il piè di pagina.
		\item \textit{makeBanner}: metodo utilizzato dalla \textit{home} per costruire il banner di presentazione del sito.
		\item \textit{listBeers}: metodo utilizzato da \textit{home} e \textit{prodotti}, costruisce la lista con le birre presenti nel sito.
		\item \textit{beerInfo}: metodo utilizzato dalla pagina di dettagli di ogni birra, costruisce la parte di dettagli, e nel caso l'utente sia loggato anche la form per l'invio di una recensione. 
		\item \textit{beerReview}: metodo utilizzato dalla pagina di dettagli di ogni birra, costruisce la lista di recensioni.
		\item \textit{makeNotfound}: costruisce una pagina di errore nel caso in cui il contenuto cercato dall'utente non sia presente nel sito.
		\item \textit{makeDeleteAccount}: costruisce la pagina che viene restituita quando si elimina un account.
		\item \textit{makeAccessdenied}: costruisce una pagina di errore nel caso in cui l'utente non abbia i permessi di vedere una determinata pagina.
		\item \textit{userInfo}: metodo utilizzato da \textit{dettagli account}, costruisce una lista con tutte le informazioni dell'account in cui si é loggati.
	\end{itemize}
\item \textbf{dbConnection.php:} si occupa dell'interazione tra il sito e il database, possiede delle costanti contenenti i dati per stabilire la connessione (username, password, host e nome del database) ed é composta dai seguenti metodi:
	\begin{itemize}
		\item \textit{openDBConnection}: apre la connessione con il database, é un metodo privato che viene richiamato solo dalle altre funzioni della classe. Nel caso di errore lancia un'eccezione;
		\item \textit{collapse}: viene utilizzato per tutte le interrogazione al database che restituiscono un risultato di una sola colonna, utilizzando questo metodo l'input fornito in forma di lista di array viene compresso in un'array monodimensionale. \'E un metodo privato che può essere richiamato solamente da altri metodi della classe;
		\item \textit{query}: metodo utilizzato per interrogare il database, accetta come parametro la query da inviare ed un valore booleano da impostare a \textit{true} nel caso in cui si voglia utilizzare il metodo collapse. Questo metodo apre la connessione al database utilizzando openDBConnection, invia la query e nel caso di errori lancia un'eccezione; se il booleano é posto a \textit{true} utilizza il metodo \textit{collapse} per comprimere il risultato della query. Infine chiude la connessione al database e restituisce i dati ricevuti tramite l'interrogazione.
		\item \textit{command}: metodo utilizzato per inviare al database comandi (insert, update, delete, alter, drop, create), l'unico parametro accettato è il comando da inviare. Questo metodo apre la connessione al database utilizzando openDBConnection, invia il comando, nel caso di errori lancia diverse eccezioni ed infine chiude la connessione.
	\end{itemize}
\end{itemize}
Per ogni pagina del sito viene prelevata la struttura dal suo file HTML, controllata l'inizializzazione e la valorizzazione della variabile dell'età, successivamente viene controllata la variabile di sessione relativa all'account nel caso in cui quella determinata pagina abbia comportamenti diversi a seconda del tipo di utente che la visualizza. Nelle pagine raggiungibili da più fonti, viene verificata la pagina di provenienza dell'utente e quindi controllata la presenza di parametri \textit{GET} o \textit{POST} nella richiesta. Infine con una serie di \textit{str\textunderscore replace()} viene aggiornato lo scheletro della pagina inizialmente caricato con i vari metodi presenti in \textit{htmlMaker.php}; solo a questo punto la pagina è pronta per la visualizzazione.