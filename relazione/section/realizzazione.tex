\section{Realizzazione}
Ogni pagina ha un suo scheletro in un file html, il quale viene aperto dal file nome\textunderscore pagina.php tramite la funzione file_get_contents(), successivamente la pagina viene popolata dalla classe php htmlMaker da noi sviluppata e solo a questo punto viene mostrata all'utente. 
Cosí facendo riusciamo a tenere totalmente separato il comportamento dalla struttura, riusciamo a gestire le sessioni e non abbiamo bisogno di fare chiamate AJAX dalle pagine.
\subsection{HTML}
Abbiamo cominciato scrivendo un header e un footer che fosseró unici per tutte le pagine, questi due file html vengono letti dalla classe htmlMaker e inseriti in tutte le pagine presenti nel sito. Successivamente abbiamo iniziato a scrivere l'html di ogni altra pagina, a partire dalla home, prodotti e contatti. Man mano che le pagine prendevano forma andavamo a spostare l'html scritto, nella classe php responsabile della generazione del contenuto delle varie pagine.
\subsection{CSS}
Per il css abbiamo lavorato parallelamente allo sviluppo delle pagine, abbiamo prima sviluppato il css principale e quando il lavoro era quasi terminato, almeno nel contenuto, abbiamo sistemato la vista da mobile, andando ad aggiungere alcune regole nel css per il mobile. Alla fine di tutto abbiamo provato a visualizzare il sito in versione stampa e abbiamo aggiunto alcune regole nel css di stampa, come per esempio la rimozione del footer, della navbar e di altre parti del sito inutili per la stampa.
\subsection{Javascript}
\subsection{PHP}
Nel nostro sito, ogni pagina ha un suo file php, inoltre abbiamo creato due classi con metodi statici, così da non dover istanziare oggetti per richiamarne le funzioni:
\begin{itemize}
\item \textbf{htmlMaker.php:}
\item \textbf{dbConnection.php:}
\end{itemize}