\section{Introduzione}
\subsection{Abstract}
\textit{La Tana del Luppolo} è un sito che nasce come vetrina per un piccolo birrificio artigianale. L'idea é quella che i clienti possano lasciare delle recensioni e dare dei voti alle birre che assaggiano, ed eventualmente acquistano durante le degustazioni organizzate periodicamente dall'azienda.
\subsection{Analisi dell'utenza}
Il sito web si rivolge ad un utenza molto variegata, prevalentemente persone appassionate di birra o alla ricerca di nuove passioni. Tale utenza comprende dai più giovani, molto abili e intuitivi nella navigazione, fino agli utenti più anziani, in genere con meno dimestichezza nel navigare su internet.

\subsection{Funzionalità}
Per accedere al sito, data la natura dei contenuti, viene predisposta una pagina utile ad assicurarsi che gli utenti siano maggiorenni e ad impedire quindi di proseguire in caso contrario. Superata la verifica dell'età viene quindi visualizzata una vetrina di birre consigliate e l'utente può procedere richiedendone dei dettagli più approfonditi, visualizzando la lista completa delle birre disponibili, accedendo al sito oppure visualizzando i contatti dell'azienda.\\ \\
La registrazione al sito permette di aggiungere e rimuovere recensioni personali delle birre, composte da una descrizione e da un voto in decimi.\\
Gli account amministratori, oltre a gestire le proprie recensioni, hanno la possibilità di rimuovere tutte le recensioni in modo da poter svolgere la loro funzione di moderazione.\\
Gli account possono essere elevati al ruolo di admin dagli amministratori del database, che si occupano inoltre dell'inserimento di nuove birre.

