\section{Accessibilitá}

\subsection{Controlli}
I controlli sono correttamente assegnati ad una label e nelle form sono raggruppati da un tag \textit{fieldset}, descritto da un rispettivo tag \textit{legend}.\\
Tramite l'attributo \textit{tabindex} l'ordine di focus tramite tab viene corretto o nascosto ad alcuni elementi di presentazione, come ad esempio le icona degli utenti delle recensioni, che sono quindi marcate con il valore \textit{presentation} dell'attributo \textit{role}.\\
I valori provenienti dai controlli prima di essere processati lato server vengono validati e sanitizzati.

\subsection{Tag vari}
I tag di heading sono stati utilizzati seguendo la corretta gerarchia ed in modo coerentemente per segnalare titoli e contenuti delle sezioni.\\
Nel sito sono assenti link circolari e sono presenti un \textit{hidden link} utile allo screen reader per saltare al contenuto della pagina e un \textit{back-to-top-button} per tornare all'inizio della pagina.\\
In aiuto ai sistemi di screen-reading sono inoltre segnalate le parole inglesi dagli attributi \textit{xml:lang} e \textit{lang} e le immagini principali sono dotate di un attributo \textit{alternative text} consono.

\subsection{Validazioni}


% ff extention WCAG Color Checker https://addons.mozilla.org/it/firefox/addon/wcag-contrast-checker
\paragraph{WCAG Color Checker}\ap{1}
L'estensione per il browser Firefox non rileva problemi tra i colori scelti ed i livelli di contrasto.

% ff extention WAVE Evaluation Tool https://addons.mozilla.org/it/firefox/addon/wave-accessibility-tool/
\paragraph{WebAIM WAVE Tool}\ap{2}
L'estensione per il browser Firefox valuta positivamente l'utilizzo dei tag, gerarchia degli heading e label dei controlli e non segnala problemi.

% Total Validator



\vfill
\section*{Links}
\begin{itemize}
    \item 1. WCAG Color Checker \url{https://addons.mozilla.org/it/firefox/addon/wcag-contrast-checker} 
    \item 2. WebAIM WAVE Tool \url{https://addons.mozilla.org/it/firefox/addon/wave-accessibility-tool/} 
\end{itemize}