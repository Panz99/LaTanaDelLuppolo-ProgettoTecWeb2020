\section{Accessibilitá}

\subsection{Controlli}
I controlli sono correttamente assegnati ad una label e all'interno di form sono raggruppati da un tag \textit{fieldset}, descritto da un rispettivo tag \textit{legend}.\\
Tramite l'attributo \textit{tabindex} l'ordine di focus tramite tasto tab viene corretto o nascosto ad alcuni elementi di presentazione, come ad esempio le icone degli utenti delle recensioni, che sono quindi marcate con il valore \textit{presentation} dell'attributo \textit{role}.\\
I valori provenienti dai controlli, prima di essere processati lato server vengono validati e sanitizzati.

\subsection{Tag vari}
I tag di heading sono stati utilizzati seguendo una corretta gerarchia ed in modo coerente per segnalare titoli e contenuti delle sezioni.\\
Nel sito sono assenti link circolari e sono presenti un \textit{hidden link} utile allo screen reader per saltare al contenuto della pagina ed un \textit{back-to-top-button} per tornare all'inizio della pagina.\\
In aiuto ai sistemi di screen-reading sono inoltre segnalate le parole inglesi dagli attributi \textit{xml:lang} e \textit{lang}, le immagini principali sono dotate di un attributo \textit{alternative text} consono e i alcuni controlli marcati con l'attributo \textit{aria-label}.\\
Tutti i tag sono chiusi in modo corretto ed ogni pagina è caratterizzata da una lista adeguata di keyword.

\subsection{Validazioni}


% ff extention WCAG Color Checker https://addons.mozilla.org/it/firefox/addon/wcag-contrast-checker
\subsubsection{WCAG Color Checker}
L'estensione\ap{1} per il browser Firefox non rileva problemi tra i colori scelti ed i livelli di contrasto.

% ff extention WAVE Evaluation Tool https://addons.mozilla.org/it/firefox/addon/wave-accessibility-tool/
\subsubsection{WebAIM WAVE Tool}
L'estensione\ap{2} per il browser Firefox non segnala problemi e valuta positivamente l'utilizzo dei tag, gerarchia degli heading e label dei controlli.

\subsubsection{Total Validator}
Tutte la pagine del sito sono state alidate attraverso l'estensione per browser di Total Validator\ap{4} e Total Validator Test\ap{3}. Il codice sorgente generato da PHP di ogni pagina è stato validato con standard \textit{XHTML5}, cosí da poter utilizzare i tag piú recenti forniti da HTML5, mantenendo una struttura in stile XHTML. Inoltre è stato verificato che le pagine garantiscano un'accessibilitá conforme agli standard \textit{WCAG21 AAA}.
\subsubsection{ChromeVox}
È stata usata l'estensione ChromeVox\ap{5} di Google Chrome per testare il funzionamento dello screen reader. ChromeVox interpreta le pagine come gli altri screen reader; è facile da usare e non ha rilevato problemi.


\vfill
\section*{Links}
\begin{enumerate}
	\item WCAG Color Checker \url{https://addons.mozilla.org/it/firefox/addon/wcag-contrast-checker} 
    \item WebAIM WAVE Tool \url{https://addons.mozilla.org/it/firefox/addon/wave-accessibility-tool/} 
    \item Total Validator Test \url{https://www.totalvalidator.com/products/index.html}
    \item Total Validator Extension \url{https://www.totalvalidator.com/products/extension.html}
    \item ChromeVox \url{https://chrome.google.com/webstore/detail/screen-reader/kgejglhpjiefppelpmljglcjbhoiplfn}
\end{enumerate}
    